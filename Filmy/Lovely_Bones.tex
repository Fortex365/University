\documentclass[a4paper, 12pt]{article}

%nastavení kódování, češtiny, fontu
\usepackage[utf8]{inputenc}
\usepackage[czech]{babel}
\usepackage[T1]{fontenc}

\usepackage{hyperref}

%Informace o dokumentu
\title{\bf Filmový seminář 1}
\author{Lukáš Netřeba}
\date{\today}

\begin{document}
\maketitle
\newpage
\tableofcontents
\newpage
\section{The Lovely Bones}
\label{LB}
The Lovely Bones (Pevné pouto), bylo dějově zasazeno do druhé poloviny minulého století což bylo skvěle reflektováno na vzhledu domů, aut, nábytku, sestřihu postav a jejich oděvů. 

Nebylo mi ihned zcela jasné, co se pod tím podklopem v kukuřičném poli stalo dokud to nebylo zřejmé v pozdějších minutách filmu. Přeci jen byla zde chybějící scéna, co s ní dotyčný vrah provedl, ona utíkala pryč, kopla ho, a poté běžela už po ulici okolo černovlasé dívky (v tento moment v tzv. \uv{mezisvětě}). Nikde nic, zda ji vrah skutečně zabil (tak jak se s tímto faktem dále ve filmu pokračovalo), jen ho kopla a vypadala to, že se ji podařilo mu utéct.

Ve filmu nazvaný \uv{mezisvět} mi dost připomněl svět z filmu {\it Stranger Things}, kde ho nazývali \uv{svět vzhůru nohama}. Svět, který je totožný s tím naším, avšak postavy z jej jsou schopny vidět do našeho klasického světa, ale postavy z klasického světa nevidí do tohoto alternativního světa s výjimkami. Jednou výjimkou byl třeba otec ztracené dívky, kdy na sebe se svoji dcerou viděli přes odraz skla. Z tohoto i usuzuji ten název filmu v češtině, Pevné pouto odvozeno od toho, že otec má natolik rád svoji dceru, že ji byl schopen vidět i do alternativního světa, kde je naživu.

Ztracená dívka v jedné ze scén přímo pro ni měla ikonickou ofinu a její výrazné modré oči, které ve mně vzbuzovali pocit, že tuto herečku musím znát z nějakého jiného filmu, či seriálu. Po pátrání jsem narazil, že v žadném seriálu ani filmu nehrála, z těch kterých jsem já viděl, ale objevila se v písničce {\it Galway girl} od {\it Eda Sheerena}.
\newpage
\section{Fantastic Fear of Everything}
\label{FFoE}

\end{document}