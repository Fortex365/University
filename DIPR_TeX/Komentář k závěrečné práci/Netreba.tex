\documentclass[a4paper, 12pt]{article}

%nastavení kódování, češtiny, fontu
\usepackage[utf8]{inputenc}
\usepackage[czech]{babel}
\usepackage[T1]{fontenc}

%Informace o dokumentu
\title{\bf Komentář k závěrečné práci}
\author{Lukáš Netřeba}
\date{\today}

\begin{document}
\maketitle
\newpage
\section{Volba závěrečné práce}
Jako závěrečnou práci jsem zvolil bakalářskou práci odevzdanou na Katedře informatiky UPOL s názvem {\bf Software na ovládání LED pásků na platformě Arduino}. Důvodem jejího vybrání byl osobní vztah s platformou Arduino a dřívějšímu vlastnímu nápadu s hraním si se zmíněnými LED pásky pomocí Arduina.

\section{Vyjádření se k této práci}
{\it Vyjádření se netýká obsahu a náplně této práce nýbrž jako tomu, jak je tato práce vypracovaná.}
\subsection{Struktura textu}
Struktura v této závěrečné práci mi přijde vhodná. Z toho důvodu, že se jedná o odborný text (bakalářskou práci) a její rozpoložení dává smysl a dodržuje pravidla takovéto práce. Konkrétně mám na mysli návaznost jednotlivých částí práce, úvod, následné seznámení se základními pojmy, teoretický návrh řešení, implementace řešení, závěr atp.
\subsection{Typografie textu}
Typografie není zrovna něco, u čeho si všímám drobných detailů, pokud nejsou pro čtenáře bezprostředně rušivé. Zvolené písmo je jedno ze standardních které se u těchto prací používá, je to navíc jedno z patkových písem, tudíž se pro čtenáře lépe čte z pohledu jednoznačnosti jednotlivých písmen. Co se čárek a teček týče, vyskytují se v nich chyby v malé míře. Ostatní prvky typografie jsou v tomto textu (bakalářská práce) dle očekávání.
\subsection{Gramatika v textu}
U gramatiky už mám co vytknout, protože jsem narazil občas na slovní spojení které nedává smysl. Konkrétně něco jako \uv{Arduino se těší,} což je přece nesmysl, Arduino se těšit ničím nemůže. Formulace vět mi přijde v pořádku, přesto místy se najde část, kde se autor mohl zkusit více zamyslet, jak své myšlenky napsat lépe. 

Pravopisná vsuvka -- ze samotné práce bylo pro čtenáře nejvíce rušivé pravopisné chyby (malé písmena u zkratky \uv{UDP} a tzv. miss-types)
\subsection{Jazyk, styl}
Jazyková a stylová stránka odpovídá bakalářské práci.
\subsection{Citace}
Citací v práci nebylo mnoho, skoro to působí, jako kdyby tam nebyla žádná. Když už tam ale byla, autor ji vhodně použil.

\end{document}