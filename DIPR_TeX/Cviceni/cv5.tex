\documentclass{article}

\usepackage[utf8]{inputenc}
\usepackage[czech]{babel}
\usepackage{blindtext}

\usepackage{amsmath}
\usepackage{amsthm}
\usepackage{amsfonts}

\usepackage{tabularx}
\usepackage{pdflscape}
\usepackage{rotating}
\usepackage[table]{xcolor}
\usepackage{xcolor,colortbl}
\usepackage{makecell}

\usepackage{listings}

\renewcommand{\lstlistlistingname}{Seznam zdrojových kódů}
\renewcommand{\lstlistingname}{Zdrojový kód}

\lstset{basicstyle=\tt, keywordstyle=\color{red!50!black}, commentstyle=\color{green!50!black},
showspaces=false, showtabs=false,
tab=\rightarrowfill,
emph={rotate},emphstyle=\underbar}

\theoremstyle{plain}
\newtheorem{veta}{Věta}

\title{Pokusný dokument pro otestování učiva DIPR}
\author{Tomáš Kühr}
\date{\today}

\begin{document}

\maketitle

\begin{abstract}
Tady bude abstrakt. \blindtext
\end{abstract}

\newpage
\tableofcontents
\newpage
\listoffigures
\listoftables
\lstlistoflistings
\newpage
\section{První sekce}\label{sec:First}

\blindtext[10]

\subsection{První podsekce}\label{subsec:Fristfirst}

\blindtext[2]

\begin{lstlisting}[language=LISP,float=t,caption={Kód třídy point}, label={lst:point}]
;; class POINT

(defclass point ()
  ((x :initform 0)
   (y :initform 0)))

(defmethod r ((point point))
  (let ((x (slot-value point 'x))
        (y (slot-value point 'y)))
    (sqrt (+ (* x x) (* y y)))))

(defmethod phi ((point point))
  (let ((x (slot-value point 'x))
        (y (slot-value point 'y)))
    (cond ((> x 0) (atan (/ y x)))
          ((< x 0) (+ pi (atan (/ y x))))
          (t (* (signum y) (/ pi 2))))))

(defmethod set-r-phi ((point point) r phi)
  (setf (slot-value point 'x) (* r (cos phi))
        (slot-value point 'y) (* r (sin phi)))
  point)

(defmethod set-r ((point point) r)
  (set-r-phi point r (phi point)))

(defmethod set-phi ((point point) phi)
  (set-r-phi point (r point) phi))

(defmethod move ((point point) dx dy)
  (setf (slot-value point 'x) (+ dx (slot-value point 'x))
        (slot-value point 'y) (+ dy (slot-value point 'y)))
  point)

(defmethod rotate ((point point) center angle)
  (let ((cx (slot-value center 'x))
        (cy (slot-value center 'y)))
    (move point (- cx) (- cy))
    (set-phi point (+ (phi point) angle))
    (move point cx cy)
    point))
\end{lstlisting}

\subsection{Druhá podsekce}
\blindtext[3]

\subsubsection{Odrážkový seznam}\label{subsubsec:odrazky}
\blindlist{itemize}[4]

\subsubsection{Seznam pojmů}\label{subsubsec:pojmy}
\blindlist{description}[10]

\section{Druhá sekce}
\blindtext[5]

\begin{table}
\centering
\caption{Popisek důležité tabulky}\label{tab:dulezita}
\begin{tabular}{|l|}
\hline
B \\
\hline
b \\
\hline
\end{tabular}
\end{table}

\subsection[Fakt delší nadpis]{Fakt dlouhá podkapitola s fakt dlouhým názvem, který se nevejde na řadek}
\blindtext[20]

\subsubsection{Krátká podkapitola}
\blindtext

\begin{veta}\label{thm:jedina}
Nechť $V$ a $W$ jsou vektorové prostory dimenzí $m$ a $n$ nad tělesem $T$ a~$M$, $N$ báze  těchto prostorů. Nechť $f$ je homomorfismus prostoru $V$ do prostoru $W$ a $A$ matice typu $m\times n$ nad tělesem $T$. Matice $A$ je maticí homomorfismu $f$ vzhledem k bázím $M$, $N$ právě tehdy, když pro každý vektor $v \in V$ je 
\begin{equation}
\langle f(v)\rangle^T_N  = A \cdot \langle v\rangle^T_M.\label{eqn:prvni}
\end{equation} 
\end{veta}

\begin{proof}
Pišme $M=\{v_1, \dots, v_m\}$, $N = \{w_1, \dots, w_n\}$ a $A=(a_{i j})$. Před\-po\-klá\-dej\-me nejprve, že $A$ je matice homomorfismu  $f$ vzhledem k bázím $M$, $N$. Pro každé $j=1, \dots, m$ je tedy
$$
f(v_j) = \sum_{i=1}^n a_{ij}w_i \,.
$$
Nechť $v$ je libovolný vektor prostoru $V$; vyjádřeme jen souřadnicemi vzhledem k bázi $M$:
$$
\langle v \rangle_M = (b_1, \dots, b_m) \text{, tj. } v = \sum_{j=1}^m b_j v_j \text{ .}
$$
Odtud
\begin{eqnarray}
f(v) = \sum_{j=1}^m b_j f(v_j) &=&\sum_{j=1}^m b_j \left(\sum_{i=1}^n a_{ij}w_i \right) = \sum_{i=1}^n \left(  \sum_{j=1}^m a_{ij}b_j \right) \cdot w_i \text{ ,} \label{eqn:druha}\\
\langle f(v) \rangle_N &=& \left( \sum_{j=1}^m a_{1j}b_j, \dots, \sum_{j=1}^m a_{nj}b_j \right) \text{ .} \label{eqn:treti}
\end{eqnarray}

\begin{align*}
f(v) = \sum_{j=1}^m b_j f(v_j) &=\sum_{j=1}^m b_j \left(\sum_{i=1}^n a_{ij}w_i \right) = \sum_{i=1}^n \left(  \sum_{j=1}^m a_{ij}b_j \right) \cdot w_i \text{ ,} \\
\langle f(v) \rangle_N &= \left( \sum_{j=1}^m a_{1j}b_j, \dots, \sum_{j=1}^m a_{nj}b_j \right) \text{ .} 
\end{align*}

$$ \forall x \in \mathbb{R}^3$$

$$\left( \begin{array}{c} x+y  \\ y+z \\ x+z \\ x \end{array}\right) =
     \left( \begin{array}{*{3}c} 1 & 1 & 0\\ 0 & 1 & 1 \\ 1 & 0 & 1 \\ 1 & 0 & 0 \end{array}\right) 
     \cdot  \left( \begin{array}{c}  x \\ y \\ z \end{array}\right) \text{ ,}$$

\end{proof}

\paragraph{Jen odstavec}
\blindtext

\begin{landscape}
\newcolumntype{x}{m{1cm}}
\begin{table}
\caption{Rozvrh}\label{tab:rozvrh}
\begin{tabular}{|*{13}{x}|}
\hline
1 & 2 & 3 & 4 & 5 & 6 & 7 & 8 & 9 & 10 & 11 & 12 & 13 \\
\hline
 & \multicolumn{2}{|l|}{\cellcolor{red!20}DIPR} & & & & & & & \multicolumn{2}{|l|}{\cellcolor{red!20}ZPPCZ} & \multicolumn{2}{l|}{\cellcolor{red!20}DIPR}  \\
\hline &&&&&&&&&&&&\\
\hline &&&&&&&&& \multicolumn{3}{|l|}{\cellcolor{red!20}DID1}  &\\
\hline
\end{tabular}
\end{table}
\end{landscape}



\blindtext

\subsubsection{Skoro poslední}
\blindtext[5]

\begin{figure}
\centering
\includegraphics[height=6cm, width=10cm, trim=2cm 1cm 1cm 1cm, clip]{funkce}
\caption{zapeklitá funkce}\label{fig:fce}
\end{figure}

viz Tabulka \ref{tab:rozvrh} na straně \pageref{tab:rozvrh}
viz Tabulka \ref{tab:dulezita}
viz kapitola \ref{sec:First}, část \ref{subsec:Fristfirst}

viz rovnost (\ref{eqn:prvni}) v důkazu věty \ref{thm:jedina}
viz kód \ref{lst:point}

\appendix
\section{Příloha}
\blindtext[2]

\begin{table}
\centering
\caption{Méně důležitá tabulka v příloze}
\begin{tabular}{|l|}
\hline
A \\
\hline
a \\
\hline
\end{tabular}
\end{table}



\end{document}